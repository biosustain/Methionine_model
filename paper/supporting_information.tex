% Options for packages loaded elsewhere
\PassOptionsToPackage{unicode}{hyperref}
\PassOptionsToPackage{hyphens}{url}
\PassOptionsToPackage{dvipsnames,svgnames,x11names}{xcolor}
%
\documentclass[journal=asbcd6,manuscript=article,layout=traditional]{achemso}
\usepackage[version=3]{mhchem}
\newcommand*\mycommand[1]{\texttt{\emph{#1}}}



\usepackage{amsmath,amssymb}
\usepackage{iftex}
\ifPDFTeX
  \usepackage[T1]{fontenc}
  \usepackage[utf8]{inputenc}
  \usepackage{textcomp} % provide euro and other symbols
\else % if luatex or xetex
  \usepackage{unicode-math}
  \defaultfontfeatures{Scale=MatchLowercase}
  \defaultfontfeatures[\rmfamily]{Ligatures=TeX,Scale=1}
\fi
\usepackage{lmodern}
\ifPDFTeX\else  
    % xetex/luatex font selection
\fi
% Use upquote if available, for straight quotes in verbatim environments
\IfFileExists{upquote.sty}{\usepackage{upquote}}{}
\IfFileExists{microtype.sty}{% use microtype if available
  \usepackage[]{microtype}
  \UseMicrotypeSet[protrusion]{basicmath} % disable protrusion for tt fonts
}{}
\makeatletter
\@ifundefined{KOMAClassName}{% if non-KOMA class
  \IfFileExists{parskip.sty}{%
    \usepackage{parskip}
  }{% else
    \setlength{\parindent}{0pt}
    \setlength{\parskip}{6pt plus 2pt minus 1pt}}
}{% if KOMA class
  \KOMAoptions{parskip=half}}
\makeatother
\usepackage{xcolor}
\setlength{\emergencystretch}{3em} % prevent overfull lines
\setcounter{secnumdepth}{5}
% Make \paragraph and \subparagraph free-standing
\ifx\paragraph\undefined\else
  \let\oldparagraph\paragraph
  \renewcommand{\paragraph}[1]{\oldparagraph{#1}\mbox{}}
\fi
\ifx\subparagraph\undefined\else
  \let\oldsubparagraph\subparagraph
  \renewcommand{\subparagraph}[1]{\oldsubparagraph{#1}\mbox{}}
\fi


\providecommand{\tightlist}{%
  \setlength{\itemsep}{0pt}\setlength{\parskip}{0pt}}\usepackage{longtable,booktabs,array}
\usepackage{calc} % for calculating minipage widths
% Correct order of tables after \paragraph or \subparagraph
\usepackage{etoolbox}
\makeatletter
\patchcmd\longtable{\par}{\if@noskipsec\mbox{}\fi\par}{}{}
\makeatother
% Allow footnotes in longtable head/foot
\IfFileExists{footnotehyper.sty}{\usepackage{footnotehyper}}{\usepackage{footnote}}
\makesavenoteenv{longtable}
\usepackage{graphicx}
\makeatletter
\def\maxwidth{\ifdim\Gin@nat@width>\linewidth\linewidth\else\Gin@nat@width\fi}
\def\maxheight{\ifdim\Gin@nat@height>\textheight\textheight\else\Gin@nat@height\fi}
\makeatother
% Scale images if necessary, so that they will not overflow the page
% margins by default, and it is still possible to overwrite the defaults
% using explicit options in \includegraphics[width, height, ...]{}
\setkeys{Gin}{width=\maxwidth,height=\maxheight,keepaspectratio}
% Set default figure placement to htbp
\makeatletter
\def\fps@figure{htbp}
\makeatother

\renewcommand{\thepage}{S\arabic{page}}
\makeatletter
\makeatother
\makeatletter
\makeatother
\makeatletter
\@ifpackageloaded{caption}{}{\usepackage{caption}}
\AtBeginDocument{%
\ifdefined\contentsname
  \renewcommand*\contentsname{Table of contents}
\else
  \newcommand\contentsname{Table of contents}
\fi
\ifdefined\listfigurename
  \renewcommand*\listfigurename{List of Figures}
\else
  \newcommand\listfigurename{List of Figures}
\fi
\ifdefined\listtablename
  \renewcommand*\listtablename{List of Tables}
\else
  \newcommand\listtablename{List of Tables}
\fi
\ifdefined\figurename
  \renewcommand*\figurename{Figure}
\else
  \newcommand\figurename{Figure}
\fi
\ifdefined\tablename
  \renewcommand*\tablename{Table}
\else
  \newcommand\tablename{Table}
\fi
}
\@ifpackageloaded{float}{}{\usepackage{float}}
\floatstyle{ruled}
\@ifundefined{c@chapter}{\newfloat{codelisting}{h}{lop}}{\newfloat{codelisting}{h}{lop}[chapter]}
\floatname{codelisting}{Listing}
\newcommand*\listoflistings{\listof{codelisting}{List of Listings}}
\makeatother
\makeatletter
\@ifpackageloaded{caption}{}{\usepackage{caption}}
\@ifpackageloaded{subcaption}{}{\usepackage{subcaption}}
\makeatother
\makeatletter
\@ifpackageloaded{tcolorbox}{}{\usepackage[skins,breakable]{tcolorbox}}
\makeatother
\makeatletter
\@ifundefined{shadecolor}{\definecolor{shadecolor}{rgb}{.97, .97, .97}}
\makeatother
\makeatletter
\makeatother
\makeatletter
\makeatother
\ifLuaTeX
  \usepackage{selnolig}  % disable illegal ligatures
\fi
\IfFileExists{bookmark.sty}{\usepackage{bookmark}}{\usepackage{hyperref}}
\IfFileExists{xurl.sty}{\usepackage{xurl}}{} % add URL line breaks if available
\urlstyle{same} % disable monospaced font for URLs
\hypersetup{
  pdftitle={Supporting information - Bayesian regression facilitates quantitative modelling of cell metabolism},
  pdfauthor={Teddy Groves; Nicholas Luke Cowie; Lars Keld Nielsen},
  colorlinks=true,
  linkcolor={blue},
  filecolor={Maroon},
  citecolor={Blue},
  urlcolor={Blue},
  pdfcreator={LaTeX via pandoc}}

\author{Teddy Groves}
\affiliation{The Novo Nordisk Foundation Center for
Biosustainability, DTU, Kongens Lyngby 2800, Denmark}

\altaffiliation{These authors contributed equally to this research.}

\email{tedgro@biosustain.dtu.dk}
\author{Nicholas Luke Cowie}
\affiliation{The Novo Nordisk Foundation Center for
Biosustainability, DTU, Kongens Lyngby 2800, Denmark}

\altaffiliation{These authors contributed equally to this research.}

\author{Lars Keld Nielsen}
\affiliation{The Novo Nordisk Foundation Center for
Biosustainability, DTU, Kongens Lyngby 2800, Denmark}
\affiliation{Australian Institute for Bioengineering and Nanotechnology
(AIBN), The University of Queensland, St Lucia 4067, Australia}





\title[]{Supporting information - Bayesian regression facilitates
quantitative modelling of cell metabolism}
\makeatletter
\begin{document}
\maketitle
\ifdefined\Shaded\renewenvironment{Shaded}{\begin{tcolorbox}[sharp corners, breakable, frame hidden, interior hidden, enhanced, boxrule=0pt, borderline west={3pt}{0pt}{shadecolor}]}{\end{tcolorbox}}\fi

\newpage{}

This document provides information in support of our article ``Bayesian
regression facilitates quantitative modelling of cell metabolism''.

The results of all reported Maud runs can be found at
\url{https://github.com/biosustain/Methionine_model/blob/main/results}.

\hypertarget{mauds-input-format}{%
\section{Maud's input format}\label{mauds-input-format}}

Maud inputs are structured directories, somewhat inspired by the PEtab
format \citep{SchmiesterSch2021}. A Maud input directory must contain a
toml \citep{preston-wernertomandgedampradyunTOMLSpecification0rc2020}
file called \texttt{config.toml} which gives the input a name,
configures how Maud will be run and tells Maud where to find the other
files, allowing these to have custom names. It must also include a file
containing a kinetic model definition, a file specifying information
about parameters and a file with information experiments. The required
structure of these files is documented at
\url{https://maud-metabolic-models.readthedocs.io/en/latest/inputting.html}.
The input is validated against a Pydantic
\citep{pydanticdevelopersPydantic2022} data model.

We chose to implement a custom input format despite the existence of
standard formats in similar areas, including SBML
\citep{keatingSBMLLevelExtensible2020} and PEtab
\citep{SchmiesterSch2021}. This choice was partly motivated by the need
to ensure flexibility as Maud was developed, but there are also features
of SBML and PEtab that make them structurally unsuitable in this
context. Our requirements for an input format included that it be
mathematics-free, so that all mathematical details are encapsulated in
source code, and that it has a detailed, verifiable structure. These
requirements made toml more attractive than SBML: toml is easier for
humans to read and edit and can straightforwardly be validated using
tools like Pydantic. Further, an SBML representation of our desired
input would not contain differential equations. It would therefore not
be interoperable with most SBML targeting software, which typically
assumes that differential equations are available and does not know
about Maud's structure.

\hypertarget{mauds-kinetic-model}{%
\section{Maud's kinetic model}\label{mauds-kinetic-model}}

\hypertarget{parameters}{%
\subsection{Parameters}\label{parameters}}

Table~\ref{tbl-params} shows all of Maud's unknown parameters along with
their dimensions

Note that Maud's metabolic model includes some quantities that are not
treated as parameters in its statistical model, including temperatures,
compartment volumes and the formation energy of water. Maud treats these
quantities as if they were known precisely: they can be configured by
the user or default values can be used. Although in practice there can
be considerable uncertainty regarding these quantities, we chose to
disregard this uncertainty in the interest of simplicity.

\hypertarget{tbl-params}{}
\begin{longtable}[]{@{}
  >{\raggedright\arraybackslash}p{(\columnwidth - 4\tabcolsep) * \real{0.2500}}
  >{\raggedright\arraybackslash}p{(\columnwidth - 4\tabcolsep) * \real{0.4722}}
  >{\raggedright\arraybackslash}p{(\columnwidth - 4\tabcolsep) * \real{0.2778}}@{}}
\caption{\label{tbl-params}Table S1 -- Parameters of Maud's statistical
model}\tabularnewline
\toprule\noalign{}
\begin{minipage}[b]{\linewidth}\raggedright
Parameter
\end{minipage} & \begin{minipage}[b]{\linewidth}\raggedright
Modelled quantity
\end{minipage} & \begin{minipage}[b]{\linewidth}\raggedright
Dimensions
\end{minipage} \\
\midrule\noalign{}
\endfirsthead
\toprule\noalign{}
\begin{minipage}[b]{\linewidth}\raggedright
Parameter
\end{minipage} & \begin{minipage}[b]{\linewidth}\raggedright
Modelled quantity
\end{minipage} & \begin{minipage}[b]{\linewidth}\raggedright
Dimensions
\end{minipage} \\
\midrule\noalign{}
\endhead
\bottomrule\noalign{}
\endlastfoot
\(\Delta_fG\) & Formation energy & metabolites \\
\(k_M\) & Michaelis Menten constants & Substrates of all
enzyme/reactions and products of reversible enzyme/reactions \\
\(k_I\) & Inhibition constants & Inhibiting metabolite/compartments of
enzyme/reactions exhibiting competitive inhibition \\
\(k_{cat}\) & Rate constants & Enzyme/reactions \\
\(L_0\) & Transfer constants & Allosteric interactions \\
\(e_T\) & T dissociation constants & Modifying metabolites of allosteric
inhibitions \\
\(e_R\) & R dissociation constants & Modifying metabolites of allosteric
activations \\
\(k_{cat\, pme}\) & Rate constants of phosphorylation modifying enzymes
& Phosphorylation modifying enzymes \\
\(v_{drain}\) & Drain fluxes & Drains, experiments \\
\(Enzyme\) & Enzyme concentrations & Enzymes, experiments \\
\(C_{unbalanced}\) & Unbalanced metabolite/compartment concentrations &
Unbalanced metabolite/compartments, experiments \\
\(C_{pme}\) & Phosphorylation modifying enzyme concentrations &
Phosphorylation modifying enzymes, experiments \\
\(\psi\) & Membrane potentials & Experiments \\
\end{longtable}

Solving the steady state problem for a given set of parameters in an
experiment yields a vector \(C_{balanced}\) of balanced metabolite
concentrations. These are combined with the balanced metabolite
concentrations \(C_{unbalanced}\) to produce a vector \(C_{mic}\) with a
concentration for each metabolite/compartment combination.

\(\Delta_fG\) parameters can optionally be fixed; this can be useful for
computational purposes, as for example to avoid estimating the formation
energy of a metabolite about which there is no available information due
to it only participating in irreversible reactions.

\hypertarget{rate-equations}{%
\subsection{Rate equations}\label{rate-equations}}

As discussed in the main text, Maud's kinetic model decomposes into
factors contributing to the flux in a metabolic network in an experiment
as shown in equation \eqref{eq-decomposition}. For succinctness, and
since Maud's model assumes that there are no interactions between
experiments, we omit any notation referring to experiments below. We
also omit any reference to the network's drain reactions: these are
modelled as being exactly determined by the values of the parameter
vector \(v_{drain}\).

\begin{equation}
F(C;\theta) = Enzyme\cdot k_{cat}\cdot Reversibility \cdot Saturation \cdot Allostery \label{eq-decomposition}
\end{equation}

The term \(Enzyme\) in equation \eqref{eq-decomposition} is a vector of
non-negative real numbers representing the concentration of the enzyme
catalysing each reaction.

The term \(k_{cat}\) in equation \eqref{eq-decomposition} is a vector of
non-negative real numbers representing the amount of flux carried per
unit of saturated enzyme.

The term \(Reversibility\) in equation \eqref{eq-decomposition} is a
vector of real numbers capturing the impact of thermodynamic effects on
the reaction's flux, as shown in equation \eqref{eq-reversibility}.

\begin{align} 
  Reversibility &= 1 - \exp(\frac{\Delta_{r}G + RT \cdot S^T \ln(C_{mic})}{RT}) \label{eq-reversibility} \\ 
  \Delta_{r}G &= S^{T}\Delta_{f}G + n F \psi \nonumber 
\end{align}

The terms in \eqref{eq-reversibility} have the following meanings:

\begin{itemize}
\tightlist
\item
  \(T\) is the temperature in Kelvin (a number),
\item
  \(R\) is the gas constant (a number),
\item
  \(\Delta_rG\) is a vector representing the Gibbs free energy change of
  each reaction in standard conditions,
\item
  \(\Delta_fG\) is a vector representing the standard condition Gibbs
  free energy change of each metabolite's formation reaction, or in
  other words each metabolite's `formation energy'.
\item
  \(n\) is a vector representing the number of charges transported by
  each reaction.
\item
  \(F\) is the Faraday constant (a number)
\item
  \(\psi\) is a vector representing each reaction's membrane potential
  (these numbers only matter for reactions that transport non-zero
  charge)
\end{itemize}

Note that, for reactions with zero transported charge, the thermodynamic
effect on each reaction is derived from metabolite formation energies.
This formulation is helpful because, provided that all reactions' rates
are calculated from the same formation energies, they are guaranteed to
be thermodynamically consistent.

The term \(n\) accounts for both the charge and the directionality. For
instance, a reaction that exports 2 protons to the extracellular space
in the forward direction would have -2 charge. If a negatively charged
molecule like acetate is exported in the forward direction, \(n\) would
be 1.

Note that this way of modelling the effect of transported charge does
not take into account that the concentration gradient used by the
transport is that of the dissociated molecules. Thus, this expression is
only correct for ions whose concentration can be expressed in the model
only in the charged form; e.g., protons, \(K^+\), \(Na^+\), \(Cl^-\),
etc.

The term \(Saturation\) in equation \eqref{eq-decomposition} is a vector
of non-negative real numbers representing, for each reaction, the
fraction of enzyme that is saturated, i.e.~bound to one of the
reaction's substrates. To describe saturation we use equation
\eqref{eq-saturation}, which is taken from
\citet{liebermeister_modular_2010} and \citet{noor_note_2013}.
Additionally, this term captures competitive inhibition: as competitive
inhibitor concentration increases, the saturation denominator increases,
effectively decreasing the saturation of the substrate on the total
enzyme pool. Conversely, as the substrate concentration increases this
term approaches 1.

\begin{align} 
Saturation_r &= a \cdot \text{free enzyme ratio}\label{eq-saturation} \\ 
a &= \prod_{\text{s substrate}}\frac{C_{mic}^s}{k_{M}^{rs}} \nonumber \\ 
\text{free enzyme ratio} &= \begin{cases} 
  \prod_{\text{s sustrate}} (1 + \frac{C_{mic}^s}{k_{M}^{rs}})^{S_sr} 
  + \sum_{\text{c inhibitor}}\frac{C_{mic}^c}{k_I^{rc}} & r\text{ irreversible} \\ 
  -1 
  + \prod_{\text{s sustrate}} (1 + \frac{C_{mic}^s}{k_{M}^{rs}})^{S_sr} 
     + \sum_{\text{c inhibitor}}\frac{C_{mic}^c}{k_I^{rc}} 
     + \prod_{\text{p product}} (1 + \frac{C_{mic}^p}{k_{M}^{rp}})^{S_pr}  & r\text{ reversible} 
    \end{cases} \nonumber
\end{align}

The term \(Allostery\) in equation \eqref{eq-decomposition} is a vector
of non-negative numbers describing the effect of allosteric regulation
on each reaction. Allosteric regulation happens when binding to a
certain molecule changes an enzyme's shape in a way that changes its
catalytic behaviour. We use equation \eqref{eq-allostery} to describe
this phenomenon, following the generalised MWC approach described in
\citet{monod_nature_1965}, \citet{changeux_2013},
\citet{popova_generalization_1975} and \citet{popova_description_1979}.

\begin{align} 
  Allostery_r &= \frac{1}{1 + L_0^r \cdot (\text{free enzyme ratio}_r \cdot \frac{Qtense}{Qrelaxed})^{subunits}}\label{eq-allostery} \\ 
       Qtense &= 1 + \sum_{\text{i inhibitor}} \frac{C_{mic}^i}{e_T^{ri}}\nonumber \\ 
     Qrelaxed &= 1 + \sum_{\text{a activator}} \frac{C_{mic}^a}{e_R^{ra}}\nonumber
\end{align}

The parameter \(L_0\) in equation \eqref{eq-decomposition} is called the
transfer constant, and the parameter vectors \(e_T\) and \(e_R\) are
called tense and relaxed dissociation constants respectively.

Finally, the term \(Phosphorylation\) in equation
\eqref{eq-decomposition} captures the important effect whereby enzyme
activity is altered due to a coupled process of phosphorylation and
dephosphorylation. This description achieves a similar behaviour to the
MWC formalism for describing allosteric regulation, but using the rates
of phosphorylation and dephosphorylation rather than concentrations of
metabolites.

\begin{align}
Phosphorylation_r &= (\frac{\alpha}{\alpha + \beta})^{subunits} \label{eq-phosphorylation}\\
\alpha &= \sum_{\text{p phosphoylator}} k_{cat\,pme}^{p} \cdot C_{pme}^p \nonumber \\
\beta &= \sum_{\text{d dephosphoylator}} k_{cat\,pme}^{d} \cdot C_{pme}^d \nonumber 
\end{align}

\hypertarget{sec-methionine-case-study}{%
\section{Methionine case study}\label{sec-methionine-case-study}}

\hypertarget{dataset-generation}{%
\subsection{Dataset generation}\label{dataset-generation}}

Starting with the model in \citet{saa_construction_2016}, we extracted
values for enzyme concentrations, boundary conditions and fluxes. We
used these values to generate MCMC samples using Maud using the priors
specified in section Section~\ref{sec-methionine-priors}. When this was
finished, we selected one sample with relatively high log probability to
use as a ground truth in our case study. These parameter values are
shown below in table Table~\ref{tbl-case-study-params}. We manually
inspected the parameter values to screen for any obviously implausible
values; we did not find any of these.

\hypertarget{sec-methionine-priors}{%
\subsection{Prior distributions compared with true parameter
values}\label{sec-methionine-priors}}

Table~\ref{tbl-case-study-params} shows the prior distributions we used
for independent parameters. The first two columns show the 1\% and 99\%
quantiles of each marginal prior distribution. True parameter value are
shown in column three, and the last column shows the z-score on log
scale of the true parameter value according the marginal prior
distribution. As can be seen from the table, there are 7 parameters for
which the true value is outside the 1\%-99\% range.

\hypertarget{tbl-case-study-params}{}
\begin{longtable}[]{@{}
  >{\raggedright\arraybackslash}p{(\columnwidth - 8\tabcolsep) * \real{0.3211}}
  >{\raggedright\arraybackslash}p{(\columnwidth - 8\tabcolsep) * \real{0.1560}}
  >{\raggedright\arraybackslash}p{(\columnwidth - 8\tabcolsep) * \real{0.1651}}
  >{\raggedright\arraybackslash}p{(\columnwidth - 8\tabcolsep) * \real{0.1101}}
  >{\raggedright\arraybackslash}p{(\columnwidth - 8\tabcolsep) * \real{0.2477}}@{}}
\caption{\label{tbl-case-study-params}Table S2 -- Parameter
specification, marginal prior distributions and true parameter values
used in our case study.}\tabularnewline
\toprule\noalign{}
\begin{minipage}[b]{\linewidth}\raggedright
parameter name
\end{minipage} & \begin{minipage}[b]{\linewidth}\raggedright
1\% prior quantile
\end{minipage} & \begin{minipage}[b]{\linewidth}\raggedright
99\% prior quantile
\end{minipage} & \begin{minipage}[b]{\linewidth}\raggedright
true value
\end{minipage} & \begin{minipage}[b]{\linewidth}\raggedright
prior Z-score of true value
\end{minipage} \\
\midrule\noalign{}
\endfirsthead
\toprule\noalign{}
\begin{minipage}[b]{\linewidth}\raggedright
parameter name
\end{minipage} & \begin{minipage}[b]{\linewidth}\raggedright
1\% prior quantile
\end{minipage} & \begin{minipage}[b]{\linewidth}\raggedright
99\% prior quantile
\end{minipage} & \begin{minipage}[b]{\linewidth}\raggedright
true value
\end{minipage} & \begin{minipage}[b]{\linewidth}\raggedright
prior Z-score of true value
\end{minipage} \\
\midrule\noalign{}
\endhead
\bottomrule\noalign{}
\endlastfoot
\(e_{𝑅}^{𝐶𝐵𝑆1,𝑎𝑚𝑒𝑡𝑐}\) & 3.430e-06 & 0.002480 & 9.3e-05 & 0.004 \\
\(e_{𝑅}^{𝐺𝑁𝑀𝑇1,𝑎𝑚𝑒𝑡𝑐}\) & 3.000e-05 & 0.002000 & 2.000e-05 & -2.787 \\
\(e_{𝑅}^{𝑀𝐴𝑇3,𝑎𝑚𝑒𝑡𝑐}\) & 1.000e-04 & 0.001000 & 3.170e-04 & 0.003 \\
\(e_{𝑅}^{𝑀𝐴𝑇3,𝑚𝑒𝑡−𝐿𝑐}\) & 4.500e-04 & 0.000800 & 6.000e-04 & 0.000 \\
\(e_{𝑅}^{𝑀𝑇𝐻𝐹𝑅1,𝑎ℎ𝑐𝑦𝑠𝑐}\) & 1.120e-07 & 0.000081 & 2.000e-06 & -0.101 \\
\(e_{𝑇}^{𝐺𝑁𝑀𝑇1,𝑚𝑙𝑡ℎ𝑓𝑐}\) & 1.120e-05 & 0.008050 & 2.290e-04 & -0.136 \\
\(e_{𝑇}^{𝑀𝑇𝐻𝐹𝑅1,𝑎𝑚𝑒𝑡𝑐}\) & 1.120e-07 & 0.000081 & 1.500e-05 &
0.549306 \\
\(k_{𝑐𝑎𝑡}^{𝐴𝐻𝐶1}\) & 1.200e+02 & 400.000000 & 2.340e+02 & 0.179861 \\
\(k_{𝑐𝑎𝑡}^{𝐵𝐻𝑀𝑇1}\) & 6.000e+00 & 35.000000 & 1.380e+01 & -0.135 \\
\(k_{𝑐𝑎𝑡}^{𝐶𝐵𝑆1}\) & 1.000e+01 & 188.000000 & 7.020e+00 & -2.887 \\
\(k_{𝑐𝑎𝑡}^{𝐺𝑁𝑀𝑇1}\) & 7.000e-01 & 60.000000 & 1.050e+01 & 0.352083 \\
\(k_{𝑐𝑎𝑡}^{𝑀𝐴𝑇1}\) & 8.200e-02 & 59.100000 & 7.900e+00 & 0.44375 \\
\(k_{𝑐𝑎𝑡}^{𝑀𝐴𝑇3}\) & 5.890e-01 & 424.000000 & 1.990e+01 & 0.080556 \\
\(k_{𝑐𝑎𝑡}^{𝑀𝐸𝑇𝐻−𝐺𝑒𝑛}\) & 4.840e-01 & 349.000000 & 1.160e+00 & -1.209 \\
\(k_{𝑐𝑎𝑡}^{𝑀𝑆1kcatMS1}\) & 1.000e+00 & 3.300000 & 1.770e+00 & -0.091 \\
\(k_{𝑐𝑎𝑡}^{𝑀𝑇𝐻𝐹𝑅1}\) & 1.300e+00 & 4.200000 & 3.170e+00 & 0.183333 \\
\(k_{𝑐𝑎𝑡}^{𝑃𝑅𝑂𝑇1}\) & 1.590e-01 & 0.222000 & 2.650e-01 & 0.41875 \\
\(k_{𝐼}^{𝐺𝑁𝑀𝑇1,𝑎ℎ𝑐𝑦𝑠𝑐}\) & 2.000e-06 & 0.001400 & 5.300e-05 & 0.010 \\
\(k_{𝐼}^{𝑀𝐴𝑇1,𝑎𝑚𝑒𝑡𝑐}\) & 3.000e-04 & 0.000400 & 3.470e-04 & 0.014 \\
\(k_{𝐼}^{𝑀𝐸𝑇𝐻−𝐺𝑒𝑛,𝑎ℎ𝑐𝑦𝑠𝑐}\) & 1.000e-06 & 0.000030 & 6.000e-06 &
0.021 \\
\(k_{𝑀}^{𝐴𝐻𝐶1,𝑎ℎ𝑐𝑦𝑠𝑐}\) & 5.220e-05 & 0.037600 & 2.320e-05 & -2.050 \\
\(k_{𝑀}^{𝐴𝐻𝐶1,𝑎𝑑𝑛𝑐}\) & 1.670e-07 & 0.000120 & 5.660e-06 & 0.081944 \\
\(k_{𝑀}^{𝐴𝐻𝐶1,ℎ𝑐𝑦𝑠−𝐿𝑐}\) & 1.580e-07 & 0.000114 & 1.060e-05 &
0.318056 \\
\(k_{𝑀}^{𝐵𝐻𝑀𝑇1,ℎ𝑐𝑦𝑠−𝐿𝑐}\) & 1.200e-05 & 0.000032 & 1.980e-05 & 0.049 \\
\(k_{𝑀}^{𝐵𝐻𝑀𝑇1,𝑔𝑙𝑦𝑏𝑐}\) & 4.720e-05 & 0.034000 & 8.460e-03 & 0.659028 \\
\(k_{𝑀}^{𝐶𝐵𝑆1,ℎ𝑐𝑦𝑠−𝐿𝑐}\) & 1.000e-06 & 0.000025 & 4.240e-05 & 3.090 \\
\(k_{𝑀}^{𝐶𝐵𝑆1,𝑠𝑒𝑟−𝐿𝑐}\) & 2.000e-06 & 0.000004 & 2.830e-06 & 0.004 \\
\(k_{𝑀}^{𝐺𝑁𝑀𝑇1,𝑎𝑚𝑒𝑡𝑐}\) & 1.300e-05 & 0.009400 & 5.200e-04 & 0.1375 \\
\(k_{𝑀}^{𝐺𝑁𝑀𝑇1,𝑎ℎ𝑐𝑦𝑠𝑐}\) & 4.100e-07 & 0.000295 & 1.100e-05 & 0.000 \\
\(k_{𝑀}^{𝐺𝑁𝑀𝑇1,𝑔𝑙𝑦𝑐}\) & 5.480e-05 & 0.039500 & 2.540e-03 & 0.189583 \\
\(k_{𝑀}^{𝐺𝑁𝑀𝑇1,𝑠𝑎𝑟𝑐𝑠𝑐}\) & 3.730e-09 & 0.000003 & 1.000e-07 & 0.000 \\
\(k_{𝑀}^{𝑀𝐴𝑇1,𝑚𝑒𝑡−𝐿𝑐}\) & 1.400e-05 & 0.000720 & 1.070e-04 & 0.074 \\
\(k_{𝑀}^{𝑀𝐴𝑇1,𝑎𝑡𝑝𝑐}\) & 5.270e-05 & 0.038000 & 2.030e-03 & 0.125694 \\
\(k_{𝑀}^{𝑀𝐴𝑇3,𝑚𝑒𝑡−𝐿𝑐}\) & 4.470e-05 & 0.032200 & 1.130e-03 & -0.029 \\
\(k_{𝑀}^{𝑀𝐴𝑇3,𝑎𝑡𝑝𝑐}\) & 5.270e-05 & 0.038000 & 2.370e-03 & 0.179167 \\
\(k_{𝑀}^{𝑀𝐸𝑇𝐻−𝐺𝑒𝑛,𝑎𝑚𝑒𝑡𝑐}\) & 7.000e-06 & 0.000013 & 9.370e-06 &
-0.135 \\
\(k_{𝑀}^{𝑀𝑆1,5𝑚𝑡ℎ𝑓𝑐}\) & 3.320e-06 & 0.002390 & 6.940e-05 & -0.124 \\
\(k_{𝑀}^{𝑀𝑆1,ℎ𝑐𝑦𝑠−𝐿𝑐}\) & 1.000e-06 & 0.000003 & 1.710e-06 & -0.054 \\
\(k_{𝑀}^{𝑀𝑇𝐻𝐹𝑅1,𝑚𝑙𝑡ℎ𝑓𝑐}\) & 7.500e-05 & 0.000088 & 8.080e-05 & -0.158 \\
\(k_{𝑀}^{𝑀𝑇𝐻𝐹𝑅1,𝑛𝑎𝑑𝑝ℎ𝑐}\) & 1.600e-05 & 0.000028 & 2.090e-05 & -0.105 \\
\(k_{𝑀}^{𝑃𝑅𝑂𝑇1,𝑚𝑒𝑡−𝐿𝑐}\) & 4.500e-05 & 0.000085 & 4.390e-05 & -2.507 \\
\(𝐿_{0}^{𝐶𝐵𝑆1}\) & 3.730e-02 & 26.800000 & 1.030e+00 & 0.017 \\
\(𝐿_{0}^{𝐺𝑁𝑀𝑇1}\) & 3.730e-02 & 26.800000 & 1.310e+02 & 0.3875 \\
\(𝐿_{0}^{𝑀𝐴𝑇3}\) & 3.730e-03 & 2.680000 & 1.080e-01 & 0.037 \\
\(𝐿_{0}^{𝑀𝑇𝐻𝐹𝑅1}\) & 1.120e-01 & 80.500000 & 3.920e-01 & -1.018 \\
\end{longtable}

\(\Delta_fG\) parameters for most metabolites were fixed; those that
were modelled as unknown had a multivariate normal prior distribution
derived from eQuilibrator
\citep{beberEQuilibratorPlatformEstimation2021}.

The values for \(\Delta_fG\) parameters, as well as all other model
parameters, can be found by inspecting the file \texttt{priors.toml}
which is online at
\url{https://github.com/biosustain/Methionine_model/blob/main/data/methionine/priors.toml}.

\hypertarget{computation}{%
\subsection{Computation}\label{computation}}

We conducted adaptive Hamiltonian Monte Carlo sampling for the full and
missing-data datasets. For the full dataset we obtained 1000 post-warmup
samples each from 4 independent Markov chains after 1000 warm-up samples
and ``hot-starting'' with a mass metric output by a previous model run.

For the missing-data dataset 250 post-warmup samples were taken from 4
indpendent Markov chains after 100 warmup samples. The sampling was
initialised using the mass matric from the complete measurement dataset
and the warmup consisted of step size adaption for 100 samples. The
resulting posterior distribution had an \(\hat{R} = 1.01\) for the
log-probability and did not exhibit post-warmup divergences that were
not a result of differential equation errors.

\hypertarget{laplace-approximation-case-study}{%
\section{Laplace approximation case
study}\label{laplace-approximation-case-study}}

To compare MCMC sampling with Laplace approximation we used a different
model with fewer parameters and state variables. This model was chosen
because we were not able to generate results for our methionine model
using Laplace approximation. The simpler case still serves to illustrate
the general issues with approximating the posterior distributions of
Bayesian kinetic models using the Laplace method, and that the
associated numerical instability is another reason to prefer other
methods where possible.

The full Maud input folders used for our Laplace approximation case
study can be found at
\url{https://github.com/biosustain/Methionine_model/tree/main/data/example_ode}
and
\url{https://github.com/biosustain/Methionine_model/tree/main/data/example_ode_laplace}.

To generate Laplace samples we used Maud's Laplace mode.

\hypertarget{multimodal-posterior-distributions}{%
\section{Multimodal posterior
distributions}\label{multimodal-posterior-distributions}}

While we have not yet observed this in practice, we expect that Maud is
capable of accurately sampling mildly multi-modal posterior
distributions, i.e.~those for which regions of parameter space with high
probability density are not very sharply separated. This is because this
ability depends primarily on the underlying inference algorithm, and
adaptive Hamiltonian Monte Carlo is known to be able to sample many such
posterior distributions. See, for example,
\citet{mangiolaSccompRobustDifferential2023}, which reports the
successful use of adaptive Hamiltonian Monte Carlo to sample posterior
distributions involving mixtures of Gaussian distributions, which are
typically multi-modal.

Moreover, in cases where it fails to sample a multi-modal posterior
distribution adaptive Hamiltonian Monte Carlo typically exhibits poor
mixing and divergent transitions, which Maud is set up to detect
automatically, making this case easy for users to diagnose.

Nonetheless, it is also well known that strongly multi-modal posterior
distributions pose a problem for adaptive Hamiltonian Monte Carlo, as
well as other MCMC algorithms. See
\citep[§3.2]{betancourtmichaelMarkovChainMonte2020} for a general
discussion of this issue. \citet{mangoubiDoesHamiltonianMonte2018}
explores the related question of the relative performance of Hamiltonian
Monte Carlo vs random walk Metropolis Hastings, finding that, for
certain cases where the target distribution exhibits highly disconnected
and similarly shaped modes, the two algorithms have similar performance.

It is therefore important to pay careful attention to MCMC diagnostics
and monitor developments in computational statistics that might expand
the range of posterior distributions that can practically be sampled.

\citet{zamora-silleroEfficientCharacterizationHighdimensional2011}
introduces the method HYPERSPACE, which is designed to identify
disconnected regions of a biochemical parameter space and characterise
them by estimating their dimensions. HYPERSPACE does not perform
parameter inference and is therefore not directly comparable with
algorithms like adaptive Hamiltonian Monte Carlo as used in Maud, but
could plausibly be adapted into such an algorithm, for example by
supplementation with importance sampling (see
\citet{vehtariParetoSmoothedImportance2022}). However, any such
algorithm would be limited by HYPERSPACE's dependency on a random walk
Metropolis Hastings algorithm. This kind of algorithm's performance is
known to scale poorly compared with Hamiltonian Monte Carlo algorithms
as the number of parameters increases: see
\citet{mangoubiDoesHamiltonianMonte2018}. Since Maud aims to fit models
with hundreds of parameters, this means that it is unlikely that
HYPERSPACE can directly be used to improve Maud's efficiency.

\hypertarget{references}{%
\section{References}\label{references}}

\renewcommand{\bibsection}{}
\bibliography{bibliography.bib}




\end{document}
